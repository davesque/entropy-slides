\documentclass{beamer}
\usetheme{metropolis}

\usepackage{enumitem}
\setlist[itemize,1]{before*=\normalsize,label=$\bullet$}
\setlist[itemize,2]{before*=\small,label=$\circ$}
\setlist[itemize,3]{before*=\footnotesize,label=\Smiley}

\usepackage{pifont}
\usepackage{marvosym}

\title{%
  Discrete Channels w/Noise \\
  \normalsize A Mathematical Theory of Communication, Part II}
\author{David Sanders}

\begin{document}

  \maketitle

  \section{First, some review...}

  \begin{frame}{Introduction}
    \begin{columns}
      \column{\dimexpr\paperwidth-10pt}
      \begin{itemize}
        \item Context of signal processing (PCM, PPM)
        \item Communication is about reproducing messages sent over channels
        \item Mathematical theory shouldn’t care about meaning of messages
        \begin{itemize}
          \item Instead, focus is on message as one selected from many possible
          \item Absence of meaning allows messages to be viewed as random variables
        \end{itemize}
        \item Any monotonic increasing function of number of possible messages
        is measure of information
        \begin{itemize}
          \item Logarithm is best choice
          \begin{itemize}
            \item Engineering parameters vary linearly w/log of possibilities
            \item More intuitive -- 2 disks have twice the information ($2^{2x}$ states)
            \item Mathematically convenient -- ``downgrades" operations; pow
            $\rightarrow$ mul; mul $\rightarrow$ add
          \end{itemize}
        \end{itemize}
      \end{itemize}
    \end{columns}
  \end{frame}

\end{document}
